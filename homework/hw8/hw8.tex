\documentclass[12pt,a4paper]{article}
\usepackage{bold-extra}
\usepackage{appendix}
\usepackage{amsfonts,amsmath,amssymb}
\usepackage{enumerate}
\usepackage{float}
\usepackage{geometry}
\usepackage{graphicx}
\usepackage{latexsym}
\usepackage{listings}
\usepackage{multicol,multirow}
\usepackage{subfigure}
\usepackage{tabularx}
\usepackage{ulem}
\usepackage{tikz}
\usepackage{xcolor}
\geometry{a4paper,left=1in,right=1in,top=1in,bottom=1in}
\begin{document}
\centerline{\Huge{{\textbf{PHYSICS I\ \ Problem Set 8}}}}
\vspace{0.5cm}
\leftline{\large{Name: Haotian Fu}}
\rightline{\large{Student ID: 520021910012}}
\paragraph{\large \textbf{Problem 1}}~{\textbf{Solution}}
\vspace{2mm}

Due to elastic collision, the kinetic energy and momentum remain constant. Suppose the
velocity of the ball is $V_1$ and the velocity of the block is $V_2$. Then

Conservation of kinetic energy
\begin{align}
    \frac{1}{2}mV_0^2 = \frac{1}{2}mV_1^2 + \frac{1}{2}(2m)V_2^2
\end{align}

Conservation of momentum
\begin{align}
    m\bar{V_0} = m\bar{V_1} + (2m)\bar{V_2}
\end{align}

Solving Eq.(1)(2)
\begin{equation}
    \left\lbrace
        \begin{aligned}
            V_1 &= -\frac{1}{3}V_0\\
            V_2 &= \frac{2}{3}V_0
        \end{aligned}
    \right.
\end{equation}

Since $V_1$ changes the sign, namely, opposite to its initial direction, the ball will
never hit the block again thus no longer having no effect on the spring. Therefore, the
energy of system of the block and spring is conserved.

Conservation of energy
\begin{align}
    \frac{1}{2}(2m)V_2^2 = \frac{1}{2}kx_{\rm max}^2
\end{align}

Therefore
\begin{align}
    x_{\rm max} = \frac{2}{3}\sqrt{\frac{2m}{k}}V_0
\end{align}

\paragraph{\large \textbf{Problem 2}}~{\textbf{Solution}}
\vspace{2mm}

The block system moves in four statges: 1) uniform linear motion; 2) simple
harmonic motion; 3) uniform linear motion. Suppose the times for each stage is $t_1,t_2,
t_3,t_4$ correspondingly.

Apparently
\begin{align}
    t_1 = t_3 = \frac{L}{V_0}
\end{align}

Then for the SHO.
\begin{align}
    \omega &= \sqrt{\frac{2k}{m}}\\
    T &= \frac{2\pi}{\omega}
\end{align}

Solving Eq.(7)(8)
\begin{align}
    t_2 = \frac{T}{2} = \pi\sqrt{\frac{m}{2k}}
\end{align}

Therefore
\begin{align}
    t_{\rm total} = t_1 + t_2 + t_3 = \frac{2L}{V_0} + \pi\sqrt{\frac{m}{2k}}
\end{align}

\paragraph{\large \textbf{Problem 3}}~{\textbf{Solution}}
\vspace{2mm}

Center of mass
\begin{align}
    \bar{r}_{\rm cm} = \frac{\sum\limits_{i=1}^3 m_i\bar{r}_i}{\sum\limits_{i=1}^3 m_i} =
    \left( \frac{7}{6},2,\frac{17}{6} \right)
\end{align}

Newton's second law
\begin{align}
    F = \sum\limits_{i=1}^3 m_i a
\end{align}

After $t=2s$
\begin{align}
    \bar{r}_{{\rm cm}_{\rm final}} = \left( \frac{669}{2},2,\frac{17}{6} \right)
\end{align}

unit: cm

\paragraph{\large \textbf{Problem 4}}~{\textbf{Solution}}
\vspace{2mm}

(a) 
\begin{align}
    m &= Al\rho\\
\Rightarrow\qquad {\rm d}m &= A\rho {\rm d}l
\end{align}

Suppose for every $i$th ${\rm d}l$, its position is $x$.

Center of mass ($x-$coordinate)
\begin{align}
    \bar{r}_{\rm cm} = \frac{\int_0^l A\rho{\rm d}l\cdot x}{\int_0^l A\rho{\rm d}l} 
    = \frac{\int_0^l A\rho l {\rm d}x}{A\rho l} = \frac{\frac{1}{2}A\rho l^2}{A\rho l}
    = \frac{1}{2}l
\end{align}

(b) Analogously, for every $i$th ${\rm d}l$, its position is $x$.
\begin{align}
    m &=  \int_0^l A\alpha x {\rm d}x = \frac{1}{2}A\alpha l^2\\
\Rightarrow\qquad {\rm d}m &= A\alpha x{\rm d}x
\end{align}


Center of mass ($x-$coordinate)
\begin{align}
    \bar{r}_{\rm cm} = \frac{\int\limits_{\rm rod} x {\rm d}m}{m} = \frac{\int_0^l x \cdot 
    A\alpha x{\rm d}x}{m}
    = \frac{\frac{1}{3}A\alpha l^3}{\frac{1}{2}A\alpha l^2} = \frac{2}{3}l
\end{align}

\paragraph{\large \textbf{Problem 5}}~{\textbf{Solution}}
\vspace{2mm}

\noindent \textbf{Method 1}

Suppose the velocity of the fisherman is $v_1$ and the velocity of the boat is $v_2$.
The distance the boat has moved is $x$.

Conservation of momentum
\begin{align}
    mv_1 + Mv_2 = 0
\end{align}

Multiple time $t$ on both sides of Eq.(22)
\begin{align}
    mS_1 + MS_2 = 0
\end{align}

where the distance the fisherman travels $S_1=l-x$ and the distance the boat moves $S_2=-x$.

Therefore
\begin{align}
    x = \frac{m}{m+M}\ l
\end{align}

\noindent \textbf{Method 2}

Suppose the original position of the fisherman is the origin and the direction the man moves
is the positive $x-$axis.

In the beginning
Center of mass
\begin{align}
    \bar{r}_{\rm cm1} = \frac{M\times \frac{l}{2}}{m+M}
\end{align}

In the end
Center of mass
\begin{align}
    \bar{r}_{\rm cm2} = \frac{m\times (l-x) + M\times (\frac{l}{2}-x)}{m+M}
\end{align}

Due to conservation of momentum, the center of mass does not move. Thus
\begin{align}
    \bar{r}_{\rm cm1} = \bar{r}_{\rm cm2}
\end{align}

Therefore
\begin{align}
    x = \frac{m}{m+M}\ l
\end{align}

\paragraph{\large \textbf{Problem 6}}~{\textbf{Solution}}
\vspace{2mm}

At time $t$, the mass of the rocket
\begin{align}
    M = m_0 - \alpha t
\end{align}

During time peirod ${\rm d}t$, the mass of gas ejected
\begin{align}
    {\rm d}m = \alpha {\rm d}t
\end{align}

Suppose the velocity of the rocket when it ejects ${\rm d}m$ gas is ${\rm d}v$, the 
constant speed of gas ejected is $u$.
\begin{align}
    u{\rm d}m = M{\rm d}v
\end{align}

Solving Eq.(27)(28)(29)
\begin{align}
    {\rm d}v = \frac{\alpha}{m_0 - \alpha t}u {\rm d}t
\end{align}

Integrate Eq.(30)
\begin{align}
    v(t) = u\ln\frac{m_0}{m_0 - \alpha t}
\end{align}

After collision, the velocity of the rockert changes its direction, Thus when the rocket stops
at the first time
\begin{align}
    |v(t)| = 2|v(T)|
\end{align}

Therefore
\begin{align}
    t = 2T - \frac{\alpha T^2}{m_0}\\
    \Delta t = T - \frac{\alpha T^2}{m_0}
\end{align}

Plug the data provided in the problem
\begin{align}
    \Delta t &= 9\ {[\rm s]}
\end{align}

\end{document}