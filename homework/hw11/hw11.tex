\documentclass[12pt,a4paper]{article}
\usepackage{bold-extra}
\usepackage{appendix}
\usepackage{amsfonts,amsmath,amssymb}
\usepackage{enumerate}
\usepackage{float}
\usepackage{geometry}
\usepackage{graphicx}
\usepackage{latexsym}
\usepackage{listings}
\usepackage{multicol,multirow}
\usepackage{subfigure}
\usepackage{tabularx}
\usepackage{extarrows}
\usepackage{ulem}
\usepackage{tikz}
\usepackage{xcolor}
\geometry{a4paper,left=1in,right=1in,top=1in,bottom=1in}
\begin{document}
\centerline{\Huge{{\textbf{PHYSICS I\ \ Problem Set 11}}}}
\vspace{0.5cm}
\leftline{\large{Name: Haotian Fu}}
\rightline{\large{Student ID: 520021910012}}
\section*{\large \textbf{Problem 1}}~{\textbf{Solution}}

\begin{enumerate}[(a)]
	\item According to the FBD, we have
	\begin{align}
	\left\{
		\begin{array}{l}			
				f \cdot h = F \cdot \frac{1}{2}h\\
				\mu_s(T\cos(36.9^\circ) + W) = f\\
				F = f + T \cdot \sin(36.9^\circ)
		\end{array}
	\right.
	\label{P1(a)}
	\end{align}
	
	Solving Eq. (\ref{P1(a)}), we know
	\begin{align}
	\left\{
		\begin{array}{l}			
				T = \frac{1000}{3} N\\
				f = 200 N\\
				F = 400N
		\end{array}
	\right.
	\label{P1(a)_sol}
	\end{align}
	
	\item Analogously, we list equations as follows
	\begin{align}
	\left\{
		\begin{array}{l}			
				f \cdot h = F \cdot \left( 1-\frac{4}{10} \right) h\\
				\mu_s(T\cos(36.9^\circ) + W) = f\\
				F = f + T \cdot \sin(36.9^\circ)
		\end{array}
	\right.
	\label{P1(b)}
	\end{align}
	
	Solving Eq. (\ref{P1(b)}), we know
	\begin{align}
	\left\{
		\begin{array}{l}			
				T = 750 N\\
				f = 300 N\\
				F = 750 N
		\end{array}
	\right.
	\label{P1(b)_sol}
	\end{align}
	
	\item Suppose the \textbf{critical height} is $xh$ where $x \in (0,1)$. Then we get
	\begin{align}
	\left\{
		\begin{array}{l}			
				f \cdot h = F \cdot \left( 1-x \right)h\\
				\mu_s(T\cos(36.9^\circ) + W) = f\\
				F = f + T \cdot \sin(36.9^\circ)
		\end{array}
	\right.
	\label{P1(c)}
	\end{align}
	
	Solving Eq. (\ref{P1(c)}), we get
	\begin{align}
	\left\{
		\begin{array}{l}			
				T = \frac{1000x}{5-7x} N\\
				\\
				f = \frac{600-600x}{5-7x} N\\
				\\
				F = \frac{600}{5-7x} N
		\end{array}
	\right.
	\label{P1(c)_sol}
	\end{align}
	
	Since at \textbf{critical height}, no matter how great the $F$ is, it cannot make the post slip, the denominator of our ideal "static" situation converges to 0. Namely
	\begin{align}
		h_{\rm critical} = \frac{5}{7} h \approx 0.71 h
	\end{align}	
\end{enumerate}

\section*{\large \textbf{Problem 2}}~{\textbf{Solution}}

Suppose the \textbf{young's modulus} of steel is $Y$. Then according to the definition of \textbf{Young's modulus}
\begin{align}
	Y \xlongequal{def} \frac{F/A}{\Delta L/L}
\end{align}

We denote the cross-area $A$ as
\begin{align}
	A = \frac{FL}{\Delta L \cdot Y}
	\label{P2_corss-area}
\end{align}

Moreover, we know
\begin{align}
	A = \pi R^2 = \frac{\pi D^2}{4}
	\label{P2_area}
\end{align}
where $R$ is radius of the cross-area and $D$ is the diameter of the cross-area.

Then solving Eq. (\ref{P2_corss-area}) and Eq. (\ref{P2_area}), we have
\begin{align}
	D = \sqrt{\frac{4FL}{\pi \Delta L Y}}
\end{align}
where $Y$ is the \textbf{Young's modulus} of steel.

\section*{\large \textbf{Problem 3}}~{\textbf{Solution}}

According to the definition of Young's modulus provided in this problem
\begin{align}
	Y \xlongequal{def} \frac{W/A}{\Delta l/l_0}
\end{align}

Hooke's law
\begin{align}
	F = -k\Delta l
\end{align}

In this problem
\begin{align}
	W = F
\end{align}

Therefore, the magnitutde of $k$ is
\begin{align}
	k = \frac{YA}{l_0}
\end{align}

\section*{\large \textbf{Problem 4}}~{\textbf{Solution}}

Suppose the desity of water is $\rho$. According to \textbf{Bernoulli's Equation}, for pipe C
\begin{align}
	p_{\rm atm} + \rho g h_1 = p_C + \frac{1}{2}\rho v_C ^2
\label{P4_Bernoulli's Equation_1}
\end{align}

Since C and E are in the same pipe
\begin{align}
	p_C = p_E
\label{P4_equilibrium}
\end{align}

For pipe E
\begin{align}
	p_{\rm atm} = p_E + \rho g h_2
\label{P4_Bernoulli's Equation_2}
\end{align}

For pipe D
\begin{align}
	p_{\rm atm} + \frac{1}{2}\rho v_D ^2 &= p_C + \frac{1}{2}\rho v_C ^2\\
	A_Cv_C &= A_Dv_D
\end{align}

Solving Eq. (\ref{P4_Bernoulli's Equation_1})(\ref{P4_equilibrium})(\ref{P4_Bernoulli's Equation_2}), we get
\begin{align}
	h_2 = 3h_1
\end{align}





\section*{\large \textbf{Problem 5}}~{\textbf{Solution}}

\begin{enumerate}[(a)]
\item Suppose the desity of water is $\rho$. According to \textbf{Bernoulli's Equation}
	\begin{align}
		p_{\rm atm} + \rho g h = p_{\rm atm} + \frac{1}{2}\rho v_0 ^2 + \rho g(h-y)
	\label{P5_Bernoulli's Equation}
	\end{align}
	
Then we get the initial velocity of the jet of water flowing out of the hole
\begin{align}
	v_0 = \sqrt{2gy}
\end{align}

Suppose the jet of water needs $t$ to hit the ground from the hole. List kinematic equations as follows.
\begin{align}
	\left\{
		\begin{array}{l}			
				\frac{1}{2}gt^2 = h-y\\
				v_0t = D
		\end{array}
	\right.
	\label{P5_kinematics}
	\end{align}
	
Therefore, we finally get
\begin{align}
	D = \sqrt{4y(h-y)}
\end{align}

\item Suppose the polynomial $\mathcal{P} = y(h-y)$. Apparently, to maximize $D$ is to maximize $\mathcal{P}$.

Since
\begin{align*}
	\dot{\mathcal{P}} = -2y + h
\end{align*}
we deduce that if and only if $y = \frac{1}{2}h$, $\mathcal{P}$ reaches its maximum.

Therefore, the hole should be placed at a depth $y = h/2$ for the jet to cover a maximum horizontal distance
\end{enumerate}


\end{document}